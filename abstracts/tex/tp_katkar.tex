
        \begin{abstract}{Multiscale Modeling of Actin Filaments }{%
            H. Katkar}{%
            IIT Kanpur, India}{%
            \IStag}
        Actin is an important protein in the cellular cytoskeleton, which polymerizes into polar fil- aments that form dynamic actin networks including filopodia, made of parallel actin filaments bundled together. Each polar filament grows at its barbed end and shrinks at its pointed end under physiological conditions. The filament also ages as the nucleotide ATP bound to an actin subunit in the filament hydrolyzes and releases inorganic phosphate, modulating its mechanical properties and binding affinity towards several important actin binding proteins. The coopera- tive nature of ATP hydrolysis and phosphate release has been under debate for several decades. This work demonstrates the use of a multiscale modeling framework to gain insights into the cooperative kinetics of these reactions. Implications of molecular level cooperativity on large scale evolution of actin filaments will be discussed. Further, the role of Enabled/vasodilator- stimulated phosphoprotein (Ena/VASP), an actin binding protein that assists filopodia forma- tion by continuously associating with the growing barbed ends of predominantly short actin filaments in the bundles, will be discussed. The structure of a wild-type Ena/VASP loosely resembles a four-arm star polymer, with each arm consisting of several important domains re- sponsible for interactions with actin. In vitro microscopy experiments of purified actin filaments with Ena/VASP mutants of varying functionalities exhibit a rich phenomenology. A model based on kinetics of individual arms of Ena/VASP is used to understand the experimentally observed binding-unbinding rates of Ena/VASP mutants on actin filaments. The modelling framework allows us to gain useful insights into dynamics of the actin network. 
        \end{abstract}
        