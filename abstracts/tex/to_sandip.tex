
    \begin{abstract_online}{The Wetting Behavior of Imidazolium Based Ionic Liquids using Molecular Dynamics Simulation}{%
        S. Bhattacharjee, \underline{S. Khan}}{%
        \IStag}{%
         Computational Nano-Science Lab, Department of Chemical & Biochemical Engineering, IIT, Patna, India}
    Ionic liquids (ILs) are of particular interest due to their tunability of physical and chemical properties and a deeper understanding of their structure-property relationship is desired. Molecular dynamics (MD) simulations are performed to understand the effect of the length of the alkyl chain attached with the cation, nature of the anions and the addition of water as aqueous medium or as impurity on the wetting behavior of imidazolium-based ILs on graphite surface. The wetting behaviors of pure or aqueous IL droplets are characterized using a density profile and orientation order parameter profile, hydrogen bond profile along the axis normal to the surface as well as through interfacial tensions.  Simulation of pure IL droplet on the surface at molecular level is very challenging compared to that of aqueous IL due to their inherent characteristics, including slow dynamics, strong coulombic interactions between cations and anions, bulky nature, and heterogeneity in structure and thus results in improper sampling, non-spherical droplet on the surface etc. Therefore, efficient methods are required to understand the wetting behavior of pure ILs properly. The wetting behavior of aqueous IL droplet depends on the distribution of IL molecules across the droplet. As there are many interfaces involved in case of a droplet on surface including liquid-vapor, solid-liquid, solid-vapor and three phase contact line, it is very important to understand the preferential adsorption of IL molecules in these interfaces. 
    
    \end{abstract_online}
    