
    \begin{abstract_online}{A Comparative Study of Tackifying Monomers to Develop Bio-Based Pressure–Sensitive Adhesive: A Computational Approach }{%
        \underline{M. Singh}, S. K. Sethi, G. Manik}{%
        }{%
        Department of Polymer and Process Engineering, IIT Roorkee, India }
    The demand of pressure sensitive adhesives (PSAs) has grown very rapidly in food packaging, stationary and healthcare industries owing to its characteristic adhesion property. Several petro- based tackifying monomers such as 2-ethylhexyl acrylate (2-EHA), iso-octyl acrylate and n-butyl acrylate provide sufficient substrate adhesion but these are neither sustainable nor renewable. The bio-sourced acrylated epoxidized linseed oil (AELO) and acrylated epoxidized methyl ester (AEME) have the potential to act as tackifying monomer which are renewable, sustainable, cost effective and ecofriendly in nature. In this investigation, molecular dynamic (MD) simulations have been employed to estimate their surface energy and binding energy against aluminum substrate. Further, mechanical response (shear and bulk modulus) was investigated to understand their resistance ability against shear deformation. The fractional free volume was estimated by varying applied stress and temperature, and further used to investigate their viscoelastic behavior to understand the ease of processability and compatibility. Besides being renewable and sustainable, both AELO and AEME showed improved surface energy, substrate binding energy, shear and bulk modulus, in comparison to chemically sourced 2-EHA. Moreover, AEME showed better performance properties than AELO, and thereby, proposed as a potential green tackifying monomer for developing PSAs. 
    
    \end{abstract_online}
    