
    \begin{abstract_online}{New Age Antimicrobial Peptides: Revealing Mode of Actions of Multi Functional AMPs using Molecular Dynamics Study}{%
        \underline{N. Desai}$^{1}$, S. Fox$^{2}$, C. Verma$^{2}$}{%
        }{%
        $^1$ Division of Biological and Life Sciences, Ahmedabad University, India\newline{}$^2$ Bio Molecular Modeling and Design, Bio informatics Institute, A-star, Singapore}
    Currently, antimicrobial resistance developed by many infectious pathogens is a severe emerging problem. Antimicrobial peptides can be used as potential alternatives to conventional antibiotics because of their multi functionality and non-specificity in targeting pathogens. To understand different mechanisms of killing via bacterial membrane by AMPs in detail and to see differences in the mode of action of two peptides, Magainin2 and Pleurocidin with different modes of action we performed Molecular Dynamics simulations. Experimentally, Magainin2 is known to form toroidal pores in the membrane whereas Pleurocidin is known to interact with the intracellular targets. Molecular dynamic simulations were run for both peptides and for each orientation for 100-1000 ns using Gromacs and the charmm36m force field. Modeling a bacterial membrane (POPE:POPG in 3:1) solvated in the TIP3P water model and 0.15M NaCl ions. \par Magainin2 was found to significantly disrupt the membrane by forming toroidal pores, however Pleurocidin also seemed to be form pores when forced in the membrane. 
    
        \textbf{References} \newline{}[1] Zhang, Ling-juan, and Richard L. Gallo. “Antimicrobial Peptides.” Current Biology 26, no. 1 (January 11, 2016): R14–19.\newline{}[2] Li, Jianguo, Jun-Jie Koh, Shouping Liu, Rajamani Lakshminarayanan, Chandra S. Verma, and Roger W. Beuerman. “Membrane Active Antimicrobial Peptides: Translating Mechanistic Insights to Design.” Frontiers in Neuroscience 11 (February 14, 2017).\newline{}[3] Scocchi, Marco, Mario Mardirossian, Giulia Runti, and Monica Benincasa. “Non-Membrane Permeabilizing Modes of Action of Antimicrobial Peptides on Bacteria.” Current Topics in Medicinal Chemistry 16, no. 1 (September 16, 2015): 76–88.
    \end{abstract_online}
    