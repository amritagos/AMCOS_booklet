
    \begin{abstract_online}{The Role of Solid-Liquid Interfacial Fluctuations in the Spontaneous Motion of Droplets}{%
        U. Saxena, S. Chouksey, \underline{K. Rane}}{%
        \IStag}{%
        Indian Institute of Technology, Gandhinagar, India}
    The density fluctuations at the solid-liquid interfaces have received great attention in recent decades in the context of protein folding. In this presentation, I will discuss them in the context of droplet motion on heterogeneous surfaces. I will first discuss our efforts to understand how the above fluctuations affect the variation of solid-liquid interfacial free energy with the nature of crystalline surface. We used the grand canonical Monte Carlo (GCMC) simulations and the cumulant expansions of the interfacial free energy to relate the above fluctuations to the interfacial entropy [1]. We observed that interfacial entropy is important for the motion of droplets when the temperature varies spatially, or temporally. I will also discuss a model system where the fluctuations are expected to strongly affect the motion of droplet [2]. Here, we used the molecular dynamics simulations to study the motion of droplet, and GCMC simulations to compute the interfacial free energies. I will end the talk by discussing our efforts to rationally design the solid surfaces having the desired solid-liquid interfacial fluctuations by using the principle of Maximum Entropy. 
    
        \textbf{References} \newline{}[1] Chouksey, S. & Rane, K. Transverse correlations near solid-liquid interface: Influence of the crystal structure of solid. Chemical Physics 517, 188–197 (2019).\newline{}[2] Saxena, U., Chouksey, S. & Rane, K. Spontaneous translation of nanodroplet over a heterogeneous surface due to thermal cycles: role of solid–liquid interfacial fluctuations. Molecular Physics 1–14 (2019). doi:10.1080/00268976.2019.1657191
    \end{abstract_online}
    