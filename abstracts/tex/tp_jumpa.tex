
        \begin{abstract}{Phase Equilibria of Binary Mixtures of Triangle-Well Fluids : Bulk vs Confined Systems}{%
            T. Chakraborti}{%
            IIT Bombay, India}{%
            \IStag}
        Confined fluids occur abundantly in nature (for example, in fault gouge particles and between micron-sized particles in colloidal suspensions [1]) and also, have numerous technological and industrial applications (such as adhesion, chromatography, membranes, microfluidic devices, lubricating fluid layers, gas adsorption, catalysis and enhanced oil recovery [2,3]). The properties of a fluid under confinement may be different from that of the bulk due to the geometric constraints imposed by the presence of solid walls as well as the additional wall-fluid interactions. Fluid phase diagrams of binary mixtures exhibit rich phase behavior including vapor-liquid equilibria (VLE) with or without azeotropes, vapor-liquid-liquid equilibria (VLLE) with heteroazeotropes and liquid-liquid equilibrium (LLE) under different conditions of temperature and pressure. In this study, [4] grand canonical transition matrix Monte Carlo simulations [5] are used to determine the fluid phase diagrams, at two different temperatures, in bulk and under confinement of weakly attractive slit pores, for four different types of binary mixtures: mixture 1 is symmetric, with a weakened interaction between the unlike species (unlike the other three mixtures), mixture 2 with asymmetry in size of the molecules, mixture 3 with asymmetry in the depth of the potential well and, finally, mixture 4 where both the size and well-depth are different. The triangle-well (TW) potential [6] is employed to model the fluid-fluid and wall-fluid interactions as this potential provides simple, qualitatively accurate and practical representation of the interactions present in real systems. The simple nature of the TW models enables us to obtain a qualitative understanding of the effects of the fluid-fluid potential parameters and the effect of confinement on the vapor-liquid coexistence behavior of the four different types of TW fluid mixtures described above. One motivation for the use of this simple potential model is that the effect of individual factors like molecular size and interaction strength can be studied in isolation, unlike experiments which provide the cumulative effect of all the parameters. Significant differences in the vapor-liquid coexistence behavior of the confined fluid from the bulk phase behavior are observed. Comparison of the bulk and confined systems indicate that the intersection of the liquid-liquid phase envelope with the vapor-liquid equilibrium (VLE) curve observed in the bulk phase of mixture 1 does not occur under confinement with the heteroazeotrope being replaced by VLE showing an azeotrope at lower pressures and liquid-liquid coexistence at higher pressure. A significant decrease in the area of the vapor + liquid coexistence region is noted for mixture 2. The supercritical behavior of component 2 is noted for the confined systems in mixtures 3 and 4 at both temperatures, however, the pressure-composition diagram in the bulk system shows supercritical behavior only at the higher temperature. Canonical Monte Carlo simulations were subsequently performed on the confined systems to obtain the energy values and the density profiles along the pore width at the equilibrium points. The latter data have been utilized to explain the phase diagrams of the confined mixtures and correlate the differences in the phase diagrams to the variation in the potential parameters. 
        \end{abstract}
        