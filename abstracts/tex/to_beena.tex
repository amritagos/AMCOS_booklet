
    \begin{abstract_online}{Skin Lipids & Their Interfaces: A Computational Approach Towards Mimicking Skin Barrier Function }{%
        B. Rai}{%
        \IStag}{%
        Tata Research Development and Design Center, India}
    Human skin is a vital organ acting as an interface between us and our surroundings. It is one of the largest organ of our body comprising of three layers – Epidermis, Dermis and Hypodermis. While epidermis is responsible for overall appearance and texture of skin, its outermost layer called stratum corneum (SC) controls its barrier function. Dermis, primarily made up of collagen and elastin, provides structural support and elasticity. The deepest layer – hypodermis, is composed of adipose tissues and provides the heat resistance.  SC is composed of 15–20 layers of flattened cells called corneocytes (Bricks) which are embedded in a lipid matrix (Mortar) composed of ceramides, cholesterol, and fatty acids. The “bricks and mortar” structure of SC makes it selectively permeable thus to protecting underlying tissue from infection, dehydration, chemicals and mechanical stress. While corneocytes of SC remain almost impermeable, ~95% of the transport across skin happens through skin lipid matrix. Hence, the main task in design of transdermal drug delivery formulations or personal care products remains as effective manipulation of skin lipids interface. Addition of specific chemical additives (permeation enhancers) in the formulation is the most common method followed. However, current industry practice to arrive at the most suitable additive is largely experimental involving trial and error in-vitro and in-vivo tests which obliviously becomes time consuming and expensive. An in-silico model, which could mimic skin barrier function at molecular scale and help screen/design permeation enhancers, will be of immense value for both pharma and cosmetics industries.    We have developed a computational model of SC which is able to mimic its barrier function. A multiscale modelling framework linking molecular (micro) to continuum (macro) scale is employed to study molecular transport across the skin. A bilayer mixture of ceramides, fatty acids and cholesterol molecule is simulated using molecular dynamics (MD) simulations. The transport properties obtained from MD simulations are further incorporated in computational fluid dynamics to compute the macroscopic transport of molecules across the skin. In this talk, I shall briefly describe this model as well illustrate its utility in product design and testing by taking a few examples from pharma and cosmetics industry. 
    
    \end{abstract_online}
    