
    \begin{abstract_online}{Mapped Averaging Method for Deriving Alternative Ensemble Averages}{%
        D. A. Kofke}{%
        \KLtag}{%
        Department of Chemical and Biological Engineering, University at Buffalo, SUNY}
    Mapped averaging is a recently published framework that provides alternative, rigorous expressions for the ensemble averages that underlie molecular simulation. [1-3]  The approach introduces knowledge from approximate theories while relying on the molecular simulation to measure the deviation from the theory. The scheme also exploits the knowledge of the forces and the Hessian of the potential. In this manner, calculation of the mapped averages by molecular simulation can proceed without contamination by noise produced by behavior that has already been captured exactly by the theory, thereby yielding results of unprecedented precision with minimal computational effort. The approach can be applied in principle to any property that can be expressed as a derivative of the free energy. It does not affect sampling, so it can be used for many properties in the same simulation. We describe our recent advances in formulating and applying these methods, focusing in particular on two broad cases: (1) the free energy, which can be obtained with great efficiency via thermodynamic integration of first-derivative properties; and (2) second-derivative properties such as the dielectric constant, which are of great practical interest, but which suffer particularly from stochastic uncertainty because their evaluation is based on averaging ensemble fluctuations when performed using conventional methods. Emphasis is placed on applications to crystalline systems, where the harmonic character of the atomic motion provides an effective baseline for the mapped average. Apart from their use in the context of molecular simulation, mapped averages may also provide a new basis for developing theoretical approaches to statistical mechanical systems. 
    
        \textbf{References} \newline{}[1] Moustafa, S.G., A.J. Schultz, and D.A. Kofke, Very fast averaging of thermal properties of crystals by molecular simulation. Phys. Rev. E, 2015. 92(4): p. 043303.\newline{}[2] Schultz, A.J., S.G. Moustafa, W. Lin, S.J. Weinstein, and D.A. Kofke, Reformulation of Ensemble Averages via Coordinate Mapping. J. Chem. Theory Comput., 2016. 12: p. 1491-1498.\newline{}[3] Schultz, A.J. and D.A. Kofke, Alternatives to conventional ensemble averages for thermodynamic properties. Current Opinion in Chemical Engineering, 2019. 23: p. 70-76.
    \end{abstract_online}
    