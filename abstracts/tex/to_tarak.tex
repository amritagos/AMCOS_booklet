
    \begin{abstract_online}{Correlation Between Glass Formation and Ion Conductivity in Polymeric Ionic Liquids }{%
        T. Patra}{%
        \IStag}{%
        Department of Chemical Engineering, IIT Madras, India}
    Polymeric ionic liquids (PILs) are promising materials to enable more environmentally stable high density energy storage devices. Realization of PILs combining high environmental and mechanical stability with maximal ion conductivity would be accelerated by an improved molecular level understanding of these materials’ structure and dynamics. It is widely recognized that both mechanical properties and ion conductivity in anhydrous PILs are intimately related to the PIL’s glass formation behavior. This represents a major challenge to the rational design of these materials, given that the basic nature of glass formation and its connection to molecular properties remains a substantial open question in polymer and condensed matter physics. Here we describe coarse-grained and atomistic molecular dynamics simulations probing the relationship between PIL architecture and interactions, glass formation behavior, and ion transport characteristics. Moreover, we identify strategies for improving ion conductivity by maximizing both PIL segmental relaxation rates and the extent of ion transport decoupling from chain dynamics. This study provides guidance towards the design of PILs with improved stability and ion conductivity for future energy applications. 
    
    \end{abstract_online}
    