
        \begin{abstract}{Vector Computing: Simulation and A.I}{%
            A. Kumar}{%
            NEC Technologies, India}{%
            \STtag}
        NEC has developed a Vector Engine (VE) for accelerated computing using vectorization, with the concept that the full application runs on the high performance Vector Engine and the operating system tasks are taken care of by the Vector Host (VH), which is a standard x86 server. This is the first time that a Vector Processor is integrated seamlessly into the Linux software environment. This allows the Vector Engine to concentrate on providing the best simulation & AI application performance. The application runs on the Vector Engine while tasks like I/O and similar OS functions get performed by the x86 CPU, taking advantage of the integration the Linux kernel. With NEC’s SX-Aurora TSUBASA VE application developer can concentrate on getting the most out of the Vector Engine and its large high speed memory engine and additionally a single core is so powerful that weak scaling applications also benefits from vectorization & show better performance on the NEC SX-Aurora TSUBASA platform than other platforms (X86 or GPU) platform. 
        \end{abstract}
        