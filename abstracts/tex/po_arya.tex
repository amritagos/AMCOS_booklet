
    \begin{abstract_online}{Molecular Dynamics Simulations on Interfacial Structure in Presence of Third Component}{%
        \underline{A. Das}$^{1, 3}$, Sk. M. Ali$^{2, 3}$}{%
        }{%
        $^1$ Nuclear Recycle Board, Bhabha Atomic Research Centre, Trombay, Mumbai, India\newline{}$^2$ Chemical Engineering Division, Bhabha Atomic Research Centre, Trombay, Mumbai, India\newline{}$^3$ Homi Bhabha National Institute, trombay, Mumbai, India}
    Microscopic understanding of the interface between two immiscible or partially miscible liquids for a biphasic system not only has an immense interest in view of mass transfer processes but also has significant technological importance in the field of science and engineering. Due to its inherent difficulty, the understanding of molecular details at liquid–liquid interface using only experimental technique is not enough to ascertain the interfacial behaviour mostly due to the fluidity of the interface and buried surroundings. The contribution of intrinsic thickness and broadening induced by capillary waves (wc) are responsible for total thickness but their determination are perchance not encountered for three components system. In this context, we have performed molecular dynamics (MD) simulations of a technologically important water–dodecane system containing tri-isoamyl phosphate (TiAP) used for reprocessing of radionuclide. MD simulations provide a microscopic view of the interfacial properties of water–dodecane/TiAP interface. Further, an empirical relation between interfacial tension and interface thickness has been established for water–dodecane/TiAP system1 (see inset of Fig. 1) which is also related by capillary wave theory (CWT) as:  
    
        \textbf{References} \newline{}[1] A. Das, Sk. M. Ali, J. Mol. Liq, 2019, 277, 217–232.\newline{}[2] D. M. Mitrinovic, A. M. Tikhonov, M. Li, Z. Huang and M. L. Schlossman, Phys. Rev. Lett. 2000, 85, 582.
    \end{abstract_online}
    