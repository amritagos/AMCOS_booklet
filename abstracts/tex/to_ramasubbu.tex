
    \begin{abstract_online}{Role of Residues at the Membrane-Water Interface in the Selectivity of Solute Transport: Computational Studies of Formate/Nitrite Transporters}{%
        R. Sankararamakrishnan}{%
        \IStag}{%
        Department of Biological Sciences and Bioengineering, IIT Kanpur, India}
    Channels and transporters are integral membrane proteins that transport solutes across the cell membranes. Many of them are selective for specific solutes. The family of Formate/Nitrite transporters (FNTs) is selective for monovalent anions [1]. The main metabolites of bacterial respiration during anaerobic mixed-acid fermentation are formate, nitrite and hydrogen sulphide. These anions become cytotoxic when accumulated in cytoplasm. Individual FNT members are involved in selectively transporting these anions. Three-dimensional structures of different FNT members indicate that they share an aquaporin-like hour-glass helical fold. Since FNTs are found only in bacteria, archaea, fungi and protists and not in mammals, they are considered as potential drug targets for many diseases caused by bacteria and fungi. Using phylogenetic analysis, we have identified eight different subgroups that include two formate, three nitrite and one hydrosulphide transporters [2]. Two subgroups, designated as YfdC-α and YfdC-β, are also recognized with unassigned function. We performed equilibrium molecular dynamics simulations and umbrella sampling on FNT members belonging to three representative FNT subfamilies. We evaluated potential of mean force (PMF) profiles of different solutes using umbrella sampling approach. Our simulation studies strongly suggest that the uncharacterized EcYfdC-α is not likely to transport monovalent anions [3]. Its physiological function is perhaps to transport neutral solutes or even cations.  
    
        \textbf{References} \newline{}[1] W. Lu et al., Biol. Chem. 394, 15-27 (2013)\newline{}[2] M. Mukherjee, M. Vajpai and R. Sankararamakrishnan, BMC Genomics 18, Art. No. 560 (2017)\newline{}[3] M. Mukherjee, A. Gupta and R. Sankararamakrishnan, Biophys. J. (2020) In press.
    \end{abstract_online}
    