
    \begin{abstract_online}{Investigation of Cholesterol Influence in Fluid Phase and Gel Phase Lipid Bilayers by Coarse-Grained Molecular Dynamic Simulation}{%
        \underline{S. Kamble}, S. Patil, A. V. R. Murthy}{%
        }{%
        Department of Applied Physics, Defence Institute of Advanced Technology, Pune, India}
    Cholesterol plays an important role to regulate various functions of the cell membrane. It is the interaction of cholesterol with lipids that influences the phase state behavior of the lipid bilayer (fluid phase and gel phase) and hence the cellular functions. Therefore, investigation on lipid-cholesterol interaction is of atmost importance. We have carried out a coarse-grained molecular dynamics simulation to understand the cholesterol influence and the phase state behavior. We chose four different lipid bilayers (DOPC, DLPC, DPPC, DSPC) and varied the cholesterol concentration. MD simulations were carried out for 12us, at constant temperature 300k and pressure 1bar using Berendsen thermostat and Barostat respectively. \par The obtained results were analyzed and found to have a strong influence on cholesterol but is different for the gel phase and fluid phase. Some of the properties were analyzed are thickness, Area per lipid, Mean square displacement and Radial distribution function It was found that thickness of the bilayer estimated after simulation is increasing with cholesterol addition by $\approx 5 \%$ for fluid phase and a decrease by $\approx 7 \%$ for gel phase. Whereas, APL found to have an overall decreasing trend by $10 \%$. At higher percentage of cholesterol, Ld phase changes to Lo phase, whereas So phase changes to Lo phase. 
    
        \textbf{References} \newline{}[1] P. Adhypak, et al., BBA Biomembrane, 1860 (2017), 253-959.\newline{}[2] M. Berkowitz., Biochimica et Biophysica Acta, 1788(2009), 86-96.\newline{}[3] M. Abraham, et al., SoftwareX, 1-2(2015), 19-25.\newline{}[4] S. Moradi, et al, RCA Advanced, (2019), 4644-4658\newline{}[5] Y.Wang,et al., Biochimica et Biophysica Acta:Biomembrane, 1858(2018), 2846-2857
    \end{abstract_online}
    