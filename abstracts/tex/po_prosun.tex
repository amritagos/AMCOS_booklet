
    \begin{abstract_online}{High Throughput Screening of Hypothetical Metal Organic Frameworks for Ethane – Ethylene Separation}{%
        \underline{P. Halder}, J. K. Singh}{%
        }{%
        Department of Chemical Engineering, IIT Kanpur, India}
    Ethylene ($C_2H_4$) is the largest feedstock in petrochemical industries, and it is produced by thermal decomposition of ethane ($C_2H_6$), in which a certain amount of $C_2H_6$ residue coexists in the product and needs to be separated to produce polymer-grade $C_2H_4$ as the feedstock for many other products, particularly the widely utilized polyethylene. The well-established industrial separation process of ethylene from ethane is cryogenic distillation, which is one of the most energy-intensive processes in the chemical industry because of the similar volatilities and sizes [1]. Hence, there is a need to find an alternative technique to minimize energy consumption. Among the technologies recently developed, adsorbent-based gas separation through pressure swing adsorption (PSA), temperature swing adsorption, or membranes is a promising technology. Currently, the database of experimentally discovered MOFs is in millions, and it is growing yearly. Hence, a hybrid approach combining machine learning algorithms with molecular simulation is necessary. \par  In this work, we screened the hypothetical Metal Organic Frameworks (h-MOF) database [2] using the molecular simulation coupled with Random Forest [3] (RF) regression algorithm. In particular, we rationalized the relation between structural and chemical properties of h-MOF with the $C_2H_6$ / $C_2H_4$ selectivity. The database consists of 137,953 hypothetical MOFs. As chemical, and structural properties are often correlated with the adsorption capacity, we calculated several structural and chemical descriptors in order to leverage the accuracy of the machine learning model. $8 \%$ h-MOFs were chosen randomly from the h-MOF dataset as a training set, and simulated at 298 K, and 1 bar, by using multi-component grand-canonical Monte Carlo (GCMC) algorithm to get the equilibrium $C_2H_6$ / $C_2H_4$ selectivity. Based on the training set the RF model was developed with 250 trees. We got 35%, and $71 \%$ accuracy in predicting $C_2H_6$ / $C_2H_4$ selectivity while taking the chemical, and structural properties as descriptors, respectively. However, using both chemical, and structural descriptors, we got 89% accuracy. Among the descriptors, void fraction (VF) played a larger role to predict the $C_2H_6$ / $C_2H_4$ selectivity. The trained RF model can reasonably predict the $C_2H_6$ / $C_2H_4$ selectivity of the remaining h-MOF materials. 
    
        \textbf{References} \newline{}[1] Liao, et al., Nature Communications, 6 (2015), 8697.\newline{}[2] Wilmer, et al., Nature Chemistry, 4 (2012), 83 – 89.\newline{}[3] Breiman, L., Machine Learning, 45 (2001), 5 – 32.
    \end{abstract_online}
    