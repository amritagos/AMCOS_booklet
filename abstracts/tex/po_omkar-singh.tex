
    \begin{abstract_online}{Characterization of Biological Water at Interface of Antimicrobial Peptide in Presence of Salts Solution}{%
        \underline{O. Singh}, D. Chakraborty}{%
        }{%
        Biophysical and Computational Chemistry laboratory, Department of Chemistry, NITK, Surathkal, Mangaluru-575025, India}
    Alkali metal ions are known to play an important role in stabilizing proteins by changing its hydration environment. Water molecules around the hydrophobic and hydrophilic sites of protein molecules are important as effect the structural and dynamics properties Hu and Jiang (2008). Molecular dynamic simulations of an antimicrobial peptide (AMP) [PDB ID: 5Z32, Mohid et al. (2019)] was carried out in presence of 0M and 0.25M salt (NaCl, KCl, LiCl) solutions in aqueous medium to characterise the interfacial water at its hydrophobic, hydrophilic and mixed region. We present here the radial distribution function (RDF), orientation profile, local structure parameter, diffusion and excess entropy of water to assess its structural and dynamics properties at the interface and bulk regions. RDF’s of Cα-Ow showed higher hydration shell in presence of Li+, Na+ and a broad peak in K+. The ions-water RDF shows stronger hydration shell in the presence of Li+, Na+ Hess and van der Vegt (2009). The density of water is found to be smaller in the presence of K+ as compare to Li+, Na+ . The diffusion coefficient of water molecules is found to be greater in case of K+ as compared to Na+ and Li+ ions. Furthermore, the distribution of the O-O-O angle of water at first solvation shell around ALA2, ARG18 and LEU11 shifted to a higher value in presence of Li+ than Na+, K+. The probability of occurrence of higher coordinated water in bulk is found to be more compared to the first and second hydration shells. The calculated excess entropy [Yan et al. (2008)] from RDF’s shows less value for Li+ compare to Na+ and K+. Water molecules around the ARG18 shows more tertrahedral order compared to the hydrophobic part of AMP indicating the presence of low density water near hydrophilic region. 
    
        \textbf{References} \newline{}[1]  Aziz, E. F., et al. (2008). “Cation-Specific Interactions with Carboxylate in Amino Acid and Acetate Aqueous \newline{}Solutions: X-ray Absorption and ab initio Calculations.” J. Phys. Chem. B, 112(40), 12567–12570.\newline{}[2] Hess, B., and Vegt, N. F. A. van der. (2009). “Cation specific binding with protein surface charges.” Proc. Natl. Acad. Sci., 106(32), 13296–13300. \newline{}[3] Hu, Z., and Jiang, J. (2008). “Molecular Dynamics Simulations for Water and Ions in Protein Crystals.” Langmuir, 24(8), 4215–4223. \newline{}[4] Mohid, S. A., et. al (2019). “Application of tungsten disulfide quantum dot-conjugated antimicrobial peptides in bio-imaging and antimicrobial therapy.” Colloids Surf. B Biointerfaces, 176, 360–370. \newline{}[5] Yan, Z., Buldyrev, S. V., and Stanley, H. E. (2008). “Relation of water anomalies to the excess entropy.” Phys. Rev. E Stat. Nonlin. Soft Matter Phys., 78(5 Pt 1), 051201.
    \end{abstract_online}
    