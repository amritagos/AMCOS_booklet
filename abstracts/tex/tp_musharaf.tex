
        \begin{abstract}{Microscopic Assessment of Liquid-Liquid Extraction System in Bulk and at the Interface}{%
            A. Das}{%
            Bhabha Atomic Research Centre}{%
            \IStag}
        The recent trend toward renewal of interest in nuclear power as a futuristic energy alternative demands a close cycle spent fuel  reprocessing  methods  with  a  long  term  vision  for  safe management of  spent  fuel.  Nuclear fuel cycle is important not only to reduce the high-active solid waste but also to produce the fresh fuel for 2nd generation nuclear reactor. The extent of reprocessing depends on the efficiency of the liquid-liquid extraction processes in which the radionuclide are separated from acidic aqueous solution. The most extensively used solvent in spent fuel reprocessing is tri-n-butyl phosphate (TBP) with n-dodecane as the diluent in the PUREX process. In spite of great success, there is a demand and continuing search for an alternative of TBP, specifically for spent fuels from fast  breeder reactor. Among many, tri-isoamyl phosphate (TIAP) has been considered as competitor which has similar chemical and radiological stability like TBP in presence of nitric acid. In this context, thermo-physical and dynamical properties of ligand-solvent systems are required to predict the extraction efficiency for a representative system which can be obtained either by performing experiments or by molecular dynamics (MD) simulations. In that perspective, MD simulations assist to arrive at a reasonable conclusion in minimum no of trials of experiments and thus reduce the cost as well as time. Furthermore, there is a continuing endeavour  at the molecular level understanding of liquid-liquid bi-phasic extraction of metal ions due to its wide level of application from pharmaceuticals to nuclear industry. Microscopic understanding of the interface between two immiscible or partially miscible liquids of any biphasic system not only has a great interest in view of mass transfer processes but also has considerable technological values in the field of chemistry, physics and biology. Due to inherent difficulty, the understanding of molecular details at liquid–liquid interface using only experimental technique is inadequate to establish the interfacial behaviour. This is mostly due to the fluidity of the interface and its concealed environment, which restricts the experimental facts. The contribution of intrinsic thickness and broadening induced by capillary waves are responsible for total thickness but the determination of these two values are perhaps not encountered for three component system. MD simulations provide a microscopic analysis of the interfacial properties of water–organic interface. The present talk will focus on evaluation of structural, thermo-physical and dynamical properties of the liquid-liquid extraction system in bulk and at the interface using molecular dynamics simulations. 
        \end{abstract}
        