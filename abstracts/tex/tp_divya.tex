
        \begin{abstract}{Microscopic View of the Crowding Effects on Hydrophobic Collapse}{%
            D. Nayar}{%
            IIT Kharagpur, India}{%
            \IStag}
        A living cell is a crowded milieu comprising of large-sized macromolecules, small co-solutes  and ions with less free water [1,2]. Crucial biological processes involving hydrophobic collapse  such as protein folding and other self-assemblies occur in this environment. However, a  molecular-level understanding of these effects remains elusive. It has been widely accepted that  these effects are induced due to size (steric) effects of crowders and the solvent excluded  volume effects that are entirely entropic in nature [1]. This excludes the role of any direct  solute-crowder or crowder-crowder attractive interactions, which however has been shown to  play a crucial role recently [3]. Therefore, molecular mechanisms associated with these effects  need to be further explored. We investigate the crowding effects of small (tri)peptides on  collapse equilibria of a generic hydrophobic polymer. Advanced molecular dynamics  simulations and statistical mechanics solvation theories are used to examine solvation  thermodynamics of polymer collapse. The unresolved role of crowder intermolecular  interactions is examined. Our results show that weak polymer-crowder attractions lead to strong  polymer collapse only at high crowder volume fractions, involving entropic depletion of  crowders from polymer surface, in accordance with the widely known depletion mechanism.  Interestingly, on making the polymer-crowder attractions stronger, polymer collapses at low  volume fractions and that too via preferential adsorption of the crowders on the polymer  surface. Strongly interacting crowders weaken the polymer collapse at high crowder volume  fractions. A transition from enthalpy-dominated to entropy-dominated polymer collapse is  observed with increasing crowder concentrations. Our results provide new insights into the  existing theories of crowding effects on macromolecular collapse and the talk will discuss its  implications on macromolecular collapse/aggregation processes.  
        \end{abstract}
        