
    \begin{abstract_online}{Accurate Computational Calorimetry of Lipid Membranes by Void-Induced Melting }{%
        R. Desikan}{%
        \IStag}{%
        Invictus Oncology Private Limited, India}
    Lamellar lipid membranes exhibit a first-order gel to liquid crystalline (fluid) phase transition which is of great relevance to biological processes and industrial applications. While molecular simulations are routinely used to accurately determine such phase transitions for simple isotropic crystalline systems, no current method accurately and quickly estimates the true transition temperature of heterogeneous two-component self-assembled interfacial systems such as lipid bilayers. Here, we present a novel computational method based on void-induced kinetic melting, and employ it to accurately estimate the gel-to-fluid phase transition temperature for a 1,2-dipalmitoyl-sn-glycero-3-phosphocholine (DPPC) bilayer from extensive fully-atomistic molecular dynamics simulations, totalling $\approx 32 \ \mu s$. The method yields a unambiguous transition temperature estimate of $T_m = 319.6 \pm 2.3$ K for DPPC bilayers, within a few Kelvin of the experimentally observed $T_m = 314.5$ K, while the previous best estimate of Tm via conventional simulated melting with the same force-field is $\approx 334$ K. Our method can be employed to assess and improve the ability of existing lipid force-fields to reproduce the gel-to-fluid phase transition, and can potentially be extended for estimating thermotropic phase transitions of multi-component lipid membranes. 
    
    \end{abstract_online}
    