
    \begin{abstract_online}{A Temperature Accelerated Sliced Sampling Study on Drug Binding/Unbinding}{%
        \underline{S. Tripathi}, N. Nair}{%
        }{%
        Department of Chemistry, IIT Kanpur, India.}
    Molecular dynamics simulations using enhanced sampling techniques are widely used to explore the rugged free energy landscapes of drug binding/unbinding in proteins. Collective variable based techniques for enhanced sampling have severe drawback that their free energy convergence is slow when the dimensionality of the landscape goes beyond two or three [1]. Sampling of large number of transverse coordinates are vital to obtain reliable free energy estimates and quick convergence. Along the extent of the reaction, the essential transverse coordinates for sampling may change. Moreover, a controlled biased sampling is essential to overcome the “leak out” of biasing energies. Recently, we proposed a novel sampling technique called “Temperature Accelerated Sliced Sampling” (TASS) to overcome these limitations [1-3]. In this work, we demonstrate the application of the method in modelling binding/unbinding reaction pathways of avibactam, an FDA approved β-lactamase inhibitor, with Class-C $\beta$-lactamase [4]. \par Based on this study we obtain full atomistic details of the binding/unbinding reactions, the intermediate steps thereof, critical interactions and free energetics. We illustrate how TASS method could aid in exploring high dimensional free energy landscapes when large number of collective variables are required, and different transverse collective coordinates are taken for different slices. Of great interest, we observe that ligand solvation/desolvation is a vital collective variable for obtaining correct free energetics of drug binding/unbinding.  
    
        \textbf{References} \newline{}[1] Awasthi, A., & Nair, N. N. WIREs Comp. Mol. Sci., 9, (2018), e1398. \newline{}[2] Awasthi, S., et al. J. Chem. Phys. 146, (2017), 094-108. \newline{}[3] Awasthi, S., et al. J. Phys. Chem. B. 122, (2018), 4299.\newline{}[4] Das, C. K., & Nair, N. N. Phys. Chem. Chem. Phys., 20, (2018), 14482-14490.
    \end{abstract_online}
    