
    \begin{abstract_online}{Understanding the Adsorption Mechanism of Arsenous Acid on Magnetite (311) Surface through Molecular Dynamics Simulations}{%
        \underline{J. Kuntail}, Y. M. Jain, A. Mukherjee, I. Sinha}{%
        }{%
        Department of Chemistry, Indian Institute of Technology (Banaras Hindu University), Varanasi, India }
    Adsorption is the favored technique for the removal of toxic arsenous acid $(As(OH)_3)$ from groundwater. The method can be made more economically workable by the utilization of attractively magnetic recyclable adsorbents. There are few reports on the successful use of magnetite as an adsorbent of arsenous acid. However, none of researcher attempt to give any systematic understanding into the molecular-level interactions between the magnetite surface and arsenous acid in the presence of water molecules. Understanding the mechanism involved is critical for the development of better adsorbents of $As(OH)_3$. In this work, we studied this issue by employing classical molecular dynamics simulations for adsorption of arsenous acid in the water on the magnetite (311) surface. Long-time radial distribution function analysis tells the two interactions (1) Fe (of magnetite) and O (of arsenous acid) and (2) O (of magnetite) and H (of arsenous acid) dominate the adsorption interactions at the molecular level. The adsorption isotherm and equilibrium constant have also been calculated and reported. 
    
        \textbf{References} \newline{}[1] Y.M. Jain et al., Ind. Eng. Chem. Res., 58(2019), 19197−19201\newline{}[2] J. Kuntail, et al., J. Mol. Liq., 288(2019), 111053−111058.\newline{}[3] A. Kyrychenko, et al., Phys. Chem. Chem. Phys, 19 (2017), 8742−8756.\newline{}[4] B. E. Monarrez-Cordero, et al., Mater. Res., 19(2016), 103−112. 
    \end{abstract_online}
    