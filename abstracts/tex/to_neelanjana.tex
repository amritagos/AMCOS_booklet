
    \begin{abstract_online}{Modulating self-assembled amyloidogenic states via solvent and temperature: Insights from Computer Simulations}{%
        N. Sengupta}{%
        \IStag}{%
        Department of Biological Sciences, Indian Institute of Science Education and Research Kolkata, India}
    Self-assembled amyloid conformations are more resistant to ‘denaturation’ than the folded states of globular proteins. This corresponds, in large part, to the nature of their conformational energy landscape. Folding is typically associated with a rough, yet ‘funnel-shaped’ energy landscape corresponding to an identifiable conformational minimum. In contrast, the landscape corresponding to amyloid formation features several equivalent minima separated from highly stable ‘amyloid’ conformations by large kinetic barriers. A recurring challenge in protein biophysics has been to identify perturbative ways to disrupt stable amyloids; consequences range from proteopathic amelioration to preservation of bio-specimens. Such goals, however, necessitate molecular-level details of the response of the amyloidogenic assemblies to perturbative conditions. Herein, I will highlight first our efforts toward understanding the effects of a cosolvent in interfering with early amyloid assembly. I will then discuss our recent work on thermal conditions affecting the stability of a putative pre-formed amyloid; the overall structural response is decoupled from the thermodynamic response of the hydrophobic core, contradicting the expectations of a folded state. Further work reveals that these effects are modulated by the hydration layer in proximity to the amyloid states; this layer contains important signatures of the overall structural and thermodynamic response. 
    
    \end{abstract_online}
    