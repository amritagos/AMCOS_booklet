
        \begin{abstract}{Multiscale Modeling Approaches In Excipient Design For Oral Drug Delivery}{%
            P. K. Jha}{%
            IIT Roorkee, India}{%
            \IStag}
        Polymeric excipients are used in controlled/modified release formulations to enhance the bioavailability of poorly soluble APIs (active pharmaceutical ingredients). The actual design of such excipients is a complicated affair, since the conventional in vitro/in vivo experiments and mechanistic models do not include the effect of excipient-API complexation and how it varies with the conditions in the gastrointestinal (GI) tract. Atomistic molecular dynamics (MD) simulations that capture the detailed excipient/API chemistry are an alluring route to develop molecular design rules of such excipients, but there is always a huge disparity between the simulated and physiological time and length scales. This talk will begin with a discussion of how we can still gain some useful insights from atomistic MD simulations of such systems. This will be followed by a discussion of systematic and generic coarse-grained approaches that can take us even further. 
        \end{abstract}
        